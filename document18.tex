\documentclass[journal,12pt,twocolumn]{IEEEtran}
%
\usepackage{setspace}
\usepackage{gensymb}
%\doublespacing
\singlespacing

%\usepackage{graphicx}
%\usepackage{amssymb}
%\usepackage{relsize}
\usepackage[cmex10]{amsmath}
%\usepackage{amsthm}
%\interdisplaylinepenalty=2500
%\savesymbol{iint}
%\usepackage{txfonts}
%\restoresymbol{TXF}{iint}
%\usepackage{wasysym}
\usepackage{amsthm}
%\usepackage{iithtlc}
\usepackage{mathrsfs}
\usepackage{txfonts}
\usepackage{stfloats}
\usepackage{bm}
\usepackage{cite}
\usepackage{cases}
\usepackage{subfig}
%\usepackage{xtab}
\usepackage{longtable}
\usepackage{multirow}
%\usepackage{algorithm}
%\usepackage{algpseudocode}
\usepackage{enumitem}
\usepackage{mathtools}
\usepackage{steinmetz}
\usepackage{tikz}
\usepackage{circuitikz}
\usepackage{verbatim}
\usepackage{tfrupee}
\usepackage[breaklinks=true]{hyperref}
%\usepackage{stmaryrd}
\usepackage{tkz-euclide} % loads  TikZ and tkz-base
%\usetkzobj{all}
\usetikzlibrary{calc,math}
\usepackage{listings}
    \usepackage{color}                                            %%
    \usepackage{array}                                            %%
    \usepackage{longtable}                                        %%
    \usepackage{calc}                                             %%
    \usepackage{multirow}                                         %%
    \usepackage{hhline}                                           %%
    \usepackage{ifthen}                                           %%
  %optionally (for landscape tables embedded in another document): %%
    \usepackage{lscape}     
\usepackage{multicol}
\usepackage{chngcntr}
%\usepackage{enumerate}

%\usepackage{wasysym}
%\newcounter{MYtempeqncnt}
\DeclareMathOperator*{\Res}{Res}
%\renewcommand{\baselinestretch}{2}
\renewcommand\thesection{\arabic{section}}
\renewcommand\thesubsection{\thesection.\arabic{subsection}}
\renewcommand\thesubsubsection{\thesubsection.\arabic{subsubsection}}

\renewcommand\thesectiondis{\arabic{section}}
\renewcommand\thesubsectiondis{\thesectiondis.\arabic{subsection}}
\renewcommand\thesubsubsectiondis{\thesubsectiondis.\arabic{subsubsection}}

% correct bad hyphenation here
\hyphenation{op-tical net-works semi-conduc-tor}
\def\inputGnumericTable{}                                 %%

\lstset{
%language=C,
frame=single, 
breaklines=true,
columns=fullflexible
}
%\lstset{
%language=tex,
%frame=single, 
%breaklines=true
%}

\begin{document}
%


\newtheorem{theorem}{Theorem}[section]
\newtheorem{problem}{Problem}
\newtheorem{proposition}{Proposition}[section]
\newtheorem{lemma}{Lemma}[section]
\newtheorem{corollary}[theorem]{Corollary}
\newtheorem{example}{Example}[section]
\newtheorem{definition}[problem]{Definition}
%\newtheorem{thm}{Theorem}[section] 
%\newtheorem{defn}[thm]{Definition}
%\newtheorem{algorithm}{Algorithm}[section]
%\newtheorem{cor}{Corollary}
\newcommand{\BEQA}{\begin{eqnarray}}
\newcommand{\EEQA}{\end{eqnarray}}
\newcommand{\define}{\stackrel{\triangle}{=}}

\bibliographystyle{IEEEtran}
%\bibliographystyle{ieeetr}


\providecommand{\mbf}{\mathbf}
\providecommand{\pr}[1]{\ensuremath{\Pr\left(#1\right)}}
\providecommand{\qfunc}[1]{\ensuremath{Q\left(#1\right)}}
\providecommand{\sbrak}[1]{\ensuremath{{}\left[#1\right]}}
\providecommand{\lsbrak}[1]{\ensuremath{{}\left[#1\right.}}
\providecommand{\rsbrak}[1]{\ensuremath{{}\left.#1\right]}}
\providecommand{\brak}[1]{\ensuremath{\left(#1\right)}}
\providecommand{\lbrak}[1]{\ensuremath{\left(#1\right.}}
\providecommand{\rbrak}[1]{\ensuremath{\left.#1\right)}}
\providecommand{\cbrak}[1]{\ensuremath{\left\{#1\right\}}}
\providecommand{\lcbrak}[1]{\ensuremath{\left\{#1\right.}}
\providecommand{\rcbrak}[1]{\ensuremath{\left.#1\right\}}}
\theoremstyle{remark}
\newtheorem{rem}{Remark}
\newcommand{\sgn}{\mathop{\mathrm{sgn}}}
\providecommand{\abs}[1]{\left\vert#1\right\vert}
\providecommand{\res}[1]{\Res\displaylimits_{#1}} 
\providecommand{\norm}[1]{\left\lVert#1\right\rVert}
%\providecommand{\norm}[1]{\lVert#1\rVert}
\providecommand{\mtx}[1]{\mathbf{#1}}
\providecommand{\mean}[1]{E\left[ #1 \right]}
\providecommand{\fourier}{\overset{\mathcal{F}}{ \rightleftharpoons}}
%\providecommand{\hilbert}{\overset{\mathcal{H}}{ \rightleftharpoons}}
\providecommand{\system}{\overset{\mathcal{H}}{ \longleftrightarrow}}
	%\newcommand{\solution}[2]{\textbf{Solution:}{#1}}
\newcommand{\solution}{\noindent \textbf{Solution: }}
\newcommand{\cosec}{\,\text{cosec}\,}
\providecommand{\dec}[2]{\ensuremath{\overset{#1}{\underset{#2}{\gtrless}}}}
\newcommand{\myvec}[1]{\ensuremath{\begin{pmatrix}#1\end{pmatrix}}}
\newcommand{\mydet}[1]{\ensuremath{\begin{vmatrix}#1\end{vmatrix}}}
%\numberwithin{equation}{section}
\numberwithin{equation}{subsection}
%\numberwithin{problem}{section}
%\numberwithin{definition}{section}
\makeatletter
\@addtoreset{figure}{problem}
\makeatother

\let\StandardTheFigure\thefigure
\let\vec\mathbf
%\renewcommand{\thefigure}{\theproblem.\arabic{figure}}
\renewcommand{\thefigure}{\theproblem}
%\setlist[enumerate,1]{before=\renewcommand\theequation{\theenumi.\arabic{equation}}
%\counterwithin{equation}{enumi}


%\renewcommand{\theequation}{\arabic{subsection}.\arabic{equation}}

\def\putbox#1#2#3{\makebox[0in][l]{\makebox[#1][l]{}\raisebox{\baselineskip}[0in][0in]{\raisebox{#2}[0in][0in]{#3}}}}
     \def\rightbox#1{\makebox[0in][r]{#1}}
     \def\centbox#1{\makebox[0in]{#1}}
     \def\topbox#1{\raisebox{-\baselineskip}[0in][0in]{#1}}
     \def\midbox#1{\raisebox{-0.5\baselineskip}[0in][0in]{#1}}

\vspace{3cm}


\title{Assignment 14}
\author{Jayati Dutta}





% make the title area
\maketitle

\newpage

%\tableofcontents

\bigskip

\renewcommand{\thefigure}{\theenumi}
\renewcommand{\thetable}{\theenumi}
%\renewcommand{\theequation}{\theenumi}


\begin{abstract}
This is a simple document explaining how to describe a linear functional on a vector space for certain conditions.
\end{abstract}

%Download all python codes 
%
%\begin{lstlisting}
%svn co https://github.com/JayatiD93/trunk/My_solution_design/codes
%\end{lstlisting}

Download all and latex-tikz codes from 
%
\begin{lstlisting}
svn co https://github.com/gadepall/school/trunk/ncert/geometry/figs
\end{lstlisting}
%


\section{Problem}
In $R^3$, let $\alpha_1 = \myvec{1\\0\\1}$, $\alpha_2 = \myvec{0\\1\\-2}$ and $\alpha_3 = \myvec{-1\\-1\\0}$. Describe explicitly a linear functional f on $R^3$ such that $f(\alpha_1)=f(\alpha_2)=0$ but $f(\alpha_3)\neq 0$
\section{Explanation}
Let us consider $\alpha = \myvec{a\\b\\c}$ such that
\begin{align}
\alpha_1 x_1 + \alpha_2 x_2 + \alpha_2 x_2 = \alpha\\
\alpha_1 x_1 + \alpha_2 x_2 + \alpha_2 x_2 = \myvec{a\\b\\c}
\end{align}
The coefficient matrix is :
\begin{align}
A = \myvec{1 & 0 & -1\\0 & 1 & -1\\1 & -2 & 0}
\end{align}
So,
\begin{align}
\myvec{1 & 0 & -1\\0 & 1 & -1\\1 & -2 & 0}\myvec{x_1\\x_2\\x_3} = \myvec{a\\b\\c}
\end{align}
Now applying row reduction to the augmented matrix, we get:
\begin{multline}
\myvec{1 & 0 & -1 & a\\0 & 1 & -1 & b\\1 & -2 & 0 & c}\xleftrightarrow[]{R_3\leftarrow R_3-R_1}\myvec{1 & 0 & -1 & a\\0 & 1 & -1 & b\\0 & -2 & 1 & c-a}\\
\xleftrightarrow[]{R_3\leftarrow R_3+2R_2}\myvec{1 & 0 & -1 & a\\0 & 1 & -1 & b\\0 & 0 & -1 & c-a+2b}\\
\xleftrightarrow[]{R_3\leftarrow R_3/(-1)}\myvec{1 & 0 & -1 & a\\0 & 1 & -1 & b\\0 & 0 & 1 & a-2b-c}\\
\xleftrightarrow[R_1\leftarrow R_1+R_3]{R_2\leftarrow R_2+R_3}\myvec{1 & 0 & 0 & 2a-2b-c\\0 & 1 & 0 & a-b-c\\0 & 0 & 1 & a-2b-c}
\end{multline}
Now,
\begin{align}
\myvec{1 & 0 & 0\\0 & 1 & 0\\0 & 0 & 1}\myvec{x_1\\x_2\\x_3} = \myvec{2a-2b-c\\a-b-c\\a-2b-c}
\end{align}
So, 
\begin{align}
x_1 = 2a-2b-c\\
x_2= a-b-c \\
x_3= a-2b-c
\end{align}
Now, as $f$ is a linear functional on $R^3$,
\begin{align}
\alpha = \alpha_1 x_1 + \alpha_2 x_2 + \alpha_2 x_2\\
\implies f(\alpha) = f(\alpha_1 x_1 + \alpha_2 x_2 + \alpha_2 x_2)\\
\implies f(\alpha) = x_1 f(\alpha_1) + x_2f(\alpha_2)+ x_3f(\alpha_3)
\end{align}
As mentioned in the problem statement, $f(\alpha_1)=f(\alpha_2)=0$ and $f(\alpha_3)\neq 0$, so let us consider $f(\alpha_3)= k$ where $k$ is a scalar constant and $k \neq 0$.

Now,
\begin{align}
f(\alpha) = x_1 f(\alpha_1) + x_2f(\alpha_2)+ x_3f(\alpha_3)\\
\implies f(\alpha) = x_3f(\alpha_3)\\
\implies f(\alpha) = x_3 k\\
\implies f(\alpha) = k (a-2b-c)\\
\implies f(a,b,c) = k (a-2b-c)
\end{align}
So, the function can be defined as:
\begin{align}
f(x,y,z) = k (x-2y-z)
\end{align}
Using this defined function, it can be verified that:
\begin{align}
f(\alpha_1) = k (x-2y-z)\\
\implies f(1,0,1)= k(1 - 2\times0 - 1)\\
\implies f(1,0,1)= 0
\end{align}
Similarly,
\begin{align}
f(\alpha_2) = k (x-2y-z)\\
\implies f(0,1,-2)= k(0 - 2\times1 + 2)\\
\implies f(0,1,-2)= 0
\end{align}
and 
\begin{align}
f(\alpha_3) = k (x-2y-z)\\
\implies f(-1,-1,0)= k(-1 + 2 - 0)\\
\implies f(-1,-1,0)= k \neq 0
\end{align}

Hence, the above problem statement is verified.
%\renewcommand{\theequation}{\theenumi}
%\begin{enumerate}[label=\thesection.\arabic*.,ref=\thesection.\theenumi]
%\numberwithin{equation}{enumi}
%\item Verification of the above problem using python code.\\
%\solution The  following Python code verifies the above solution.
%\begin{lstlisting}
%codes/multiplication_test.py
%\end{lstlisting}
%%%
%\end{enumerate}

\end{document}



